\documentclass{article} % For LaTeX2e
\usepackage{nips13submit_e,times}
\usepackage{hyperref}
\usepackage{url}
%\documentstyle[nips13submit_09,times,art10]{article} % For LaTeX 2.09


\title{Recognize My Face}


\author{

Durmus U. Karatay
Department of Physics
University of Washington
\texttt{ukaratay@uw.edu}
\And
Mathias Smith
Department of Physics
University of Washington
\texttt{mwsmith2@uw.edu}

}

\newcommand{\fix}{\marginpar{FIX}}
\newcommand{\new}{\marginpar{NEW}}

\begin{document}

\maketitle

\begin{abstract}
The project aims to utilize algorithms which can distinguish features in a set of images.  We employ Principal Component Analysis (PCA) and Linear Discriminant Analysis (LDA).  PCA allows us to identify features based on image similarities, and LDA helps distinguish between features.  
\end{abstract}

\section{Introduction}

In this work we explore a well known approach for facial recognition using PCA and LDA.  PCA treats the system as 2-dimensional and projects face images into a face-pace that has the information of the variation among the training images. The basis of this face-space is referred to as eigenfaces.  We try to identify different sets of image features by calculating the projections of eigenfaces onto the images.  However, PCA cannot differentiate well between 3-dimensional rotations and different lighting conditions.  These shortcomings limit the usefulness of PCA in facial recognition.

LDA can compensate for the shortcomings of PCA if they are used in conjunction.  LDA produces well separated classes in a low dimensional face-space.



The test set consists of about 2000 images of 20 different individuals with three different characteristic features: orientation, facial expression, inclusion of sunglasses.

\section{Methods}

\subsection{PCA}

explain math

explain results

\subsection{LDA}

explain idea

\section{Conclusion}


\end{document}